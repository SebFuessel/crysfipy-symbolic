\documentclass{article}
\usepackage{amsmath,amsthm,amssymb}
\usepackage{graphicx}
\usepackage[bookmarks=true,															
bookmarksnumbered=true,
colorlinks=true,
urlcolor=blue,
linkcolor=blue,
citecolor=blue]{hyperref}


\newcommand*{\hham}{\mathcal{H}}

\newtheorem{definition}{Definition}
\newtheorem{problem}{Problem}

\begin{document}
	
\section{Motivation}
In a project of finding analytical solutions to the problem of energy levels of electron filled orbital in the crystal field, we construct a Hamiltonian, with certain symmetries originating from the symmetry of the crystal.

With ideas of quantum mechanics the Hamiltonian can be represented as a matrix $\hham \in \mathbb{C}^n_n$, where n denotes the dimensionality of the problem.
The eigenvalues of $\hham$ correspond to the energy levels, for which we desire to have analytical solutions.
Physical symmetries of the problem will imply certain symmetries of $\hham$ matrix.
For example, see the following matrix corresponds to energy levels of $J=2$ orbital in an $222$ point group symmetry:
\begin{center}
\resizebox{0.98\linewidth}{!}{%
\begin{math}
	\label{Hcef}
	\hham_1 = 
	\left[
	\begin{matrix}
		6 B_{20} + 12 B_{40} & 0 & \sqrt{6} B_{22} + 3 \sqrt{6} B_{42} & 0 & 12 B_{44}\\
		0 & - 3 B_{20} - 48 B_{40} & 0 & 3 B_{22} - 12 B_{42} & 0\\
		\sqrt{6} B_{22} + 3 \sqrt{6} B_{42} & 0 & - 6 B_{20} + 72 B_{40} & 0 & \sqrt{6} B_{22} + 3 \sqrt{6} B_{42}\\
		0 & 3 B_{22} - 12 B_{42} & 0 & - 3 B_{20} - 48 B_{40} & 0\\
		12 B_{44} & 0 & \sqrt{6} B_{22} + 3 \sqrt{6} B_{42} & 0 & 6 B_{20} + 12 B_{40}
	\end{matrix}
	\right],
\end{math}
}
\end{center}
$B_{ij} \in \mathbb{R}$.

As a fundamental postulate of quantum mechanics $\hham$ is hermitian.
In many cases some diagonals of the $\hham$ will contain 0.
In addition, some matrices, as the one in the example, are \textit{persymmetric}.

Finding the eigenvalues of $\hham$ corresponds to finding roots of its characteristic polynomial,  $p_\lambda(\hham) = det(\hham -\lambda \mathbb{I})$.
Analytical formulas of polynomial roots exist only for irreducible polynomials of degree three, as Cardano formulas.
Given the symmetry of the Hamiltonian, even though $p_\lambda(\hham)$ is mostly of degree higher than three, it can be reduced to a form containing polynomials of degree 3 or lower.

\section{Introduction}
\begin{definition}[Persymmetric matrix]
Following \href{https://en.wikipedia.org/wiki/Persymmetric_matrix}{Wikipedia}

A persymmetric matrix is a square matrix which is symmetric with respect to the northeast-to-southwest diagonal.

Let $A=(a_{ij}) \in \mathbb{C}^n_n$. $A$ is persymmetric $\Leftrightarrow $  $\forall i,j, a_{ij} = a_{n-j+1, n-i+1}$.
\end{definition}

\begin{definition}[m-diagonal matrix]
Let $A=(a_{ij}) \in \mathbb{C}^n_n$. $A$ is m-diagonal $\Leftrightarrow$ m diagonals of $A$ contain nonzero elemnts.
\end{definition}
Example, matrix \ref{Hcef} is hermitian, persymmetric, 5-diagonal.



\section{Problem}

\begin{problem}
For a hermitian, persymmetric, m-diagonal matrix $A \in \mathbb{C}_n^n$, for which $m$ and $n$ the characteristic polynomial of $A$ can be reduced to polynomials of degree two?
\end{problem}

\begin{problem}
For a hermitian, m-diagonal matrix $A \in \mathbb{C}_n^n$, for which $m$ and $n$ the characteristic polynomial of $A$ can be reduced to polynomials of degree two?
\end{problem}
	
\end{document}
